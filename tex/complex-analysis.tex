\documentclass[thmcnt=section, color=cyan, 12pt]{my-elegantbook}

% index page
\usepackage{imakeidx}
\makeindex[columns=2, intoc, options=-s index_style.ist]

% title and author
\title{Complex Analysis}
\author{Isaac FEI}

% reference file
\addbibresource{complex-analysis.bib} 

% image of the book cover
\cover{cover}

\begin{document}

% Print title and cover page
\maketitle

%--------
% preface
%--------

\frontmatter
\chapter*{Preface}

This book mainly follows the structure of \cite{steinComplexAnalysis2003}.

%------------------------------

% Print table of contents
\tableofcontents
\mainmatter

%-------------------------------
% main document starts from here
%-------------------------------

%==============================

\chapter{Point-Set Topology}

%------------------------------

\section{Preimages}

Let $f: X \to Y$ be a function between two sets. The \textbf{preimage}\index{preimage} of a subset $S \subseteq Y$ under $f$,
written $f^{-1}(S)$, is defined by
\begin{align*}
	f^{-1}(S) := \{x \in X \mid f(x) \in S\}
\end{align*}
In other words,
\begin{align*}
	x \in f^{-1}(S) \iff f(x) \in S
\end{align*}

%==============================

\chapter{Complex Functions}

%------------------------------

\section{Holomorphic Functions}

\begin{definition}
	Let $f: \Omega \to \C$ be a function defined on an open set $\Omega \subseteq \C$.
	We say $f$ is \textbf{complex differentiable}\index{complex differentiable functions}
	at $z \in \Omega$ if the quotient
	\begin{align*}
		\frac{f(z + h) - f(z)}{h}
	\end{align*}
	converges when $h \to 0$.
	Its limit is denoted by $f^\prime(z)$
	and called the derivative of f at $z$.
	We write
	\begin{align*}
		f^\prime(z) = \lim_{h \to 0} \frac{f(z + h) - f(z)}{h}
	\end{align*}

	We say $f$ is \textbf{holomorphic in an open set}\index{holomorphic in an open set} $\Omega$
	if $f$ is complex differentiable at every point of $\Omega$.

	And we say $f$ is \textbf{holomorphic at a point}\index{holomorphic at a point} $z \in \Omega$ if $f$ is holomorphic in a neighbourhood of $z$.
\end{definition}

\begin{note}
	It should be emphasized that $f$ being holomorphic at a point $z$
	is not the same as $f$ being complex differentiable at $z$.
	Being holomorphic at a point is a stronger condition since
	it means that $f$ should be complex differentiable
	at every point in a neighbourhood of that point.
\end{note}

%==============================

\chapter{Cauchy's Theorem and Its Applications}

%------------------------------

\section{Goursat's Theorem}

\begin{theorem}[Goursat's Theorem] \label{thm:4}
	Let $\Omega \subseteq \C$ be an open set and $T \subseteq \Omega$ a triangle whose interior is also contained in $\Omega$.
	If $f$ is holomorphic in $\Omega$, then
	\begin{align*}
		\int_{T} f(z) \dif z = 0
	\end{align*}
\end{theorem}

We could state and prove Goursat's Theorem
considering the counter integral along a rectangle.
But we will see in Corollary~\ref{cor:1}
that the rectangle version is an immediate corollary of this triangle version
since we can divide a rectangle into two triangles.

\begin{proof}
	% TODO
\end{proof}

\begin{corollary} \label{cor:1}
	If $f$ is holomorphic in an open set $\Omega$
	containing a rectangle $R$ and its interior, then
	\begin{align*}
		\int_{R} f(z) \dif z = 0
	\end{align*}
\end{corollary}

%------------------------------

\section{Local Existence of Primitives and Cauchy's Theorem in a Disk}

\begin{theorem}
	If $f$ is holomorphic in an open disk $D \subseteq \C$, then $f$ has a primitive there.
\end{theorem}

\begin{proof}
	Without loss of generality, we may assume the disk $D$ is centered at the origin.
	If it is centered at $z_0$,
	we can translate the disk to the origin and define $f(z) = f(z + z_0)$.
	Then we find a primitive for $f$, say $F$.
	The primitive for $f$ is then given by $F(z) = F(z - z_0)$.

	Assume $D$ is centered at the origin,
	we are going to construct a primitive $F$ for $f$ in $D$
	using a contour integral.
	For any point $z \in D$,
	curve $\gamma_z$ is taken to be the horizontal line segment
	from $0$ to $\RE z$
	followed by the vertical segment from $\RE z$ to $z$.
	Note that $\gamma_z$ is indeed contained in $D$.
	Let function $F$ be defined by
	\begin{align*}
		F(z) = \int_{\gamma_z} f(w) \dif w
	\end{align*}
	We will show $F$ is indeed a primitive for $f$ in $D$.
	Hence, we need to estimate the quotient $\frac{F(z+h) - F(z)}{h}$ when $h \to 0$.

	This invites us to consider the contour integral of $f$ along curve $\gamma_{z_h} - \gamma_z$. As illustrated in Figure~\ref{fig:1},
	by adding a rectangle and a triangle without change the value of the contour integral
	due to Theorem~\ref{thm:4} and Corollary~\ref{cor:1}, one may verify that
	\begin{align}
		F(z+h) - F(z)
		= \int_{\gamma_{z+h} - \gamma_z} f(w) \dif w
		= \int_\eta f(w) \dif w
		\label{eq:6}
	\end{align}
	where $\eta$ is the line segment from $z$ to $z+h$.

	\begin{figure}[H]
		\centering
		\includegraphics[width=0.7\textwidth]{figures/adding-a-rectangle-and-a-triangle.png}
		\caption{Construction of curve $\eta$ by adding a rectangle and a triangle to the curve $\gamma_{z+h} - \gamma_z$.}
		\label{fig:1}
	\end{figure}

	Now, we are going to estimate $f(w)$ about the point $z$. Because $f$ is continuous, we can write
	\begin{align*}
		f(w) = f(z) + \psi(w), \quad w \in D \setminus \{z\}
	\end{align*}
	where $\psi(w)$ is continuous and
	\begin{align*}
		\lim_{w \to z} \psi(w) = 0
	\end{align*}

	\begin{note}
		Estimating $f(w)$ about $z$ using the derivative of $f$ at $z$,
		like what we did in the proof of Theorem~\ref{thm:4}, may be an overkill.
		It suffices to estimate $f(w)$ by exploiting the continuity of $f$.
		We have already used the fact that $f$ is holomorphic to construct
		the curve $\eta$ by applying the Goursat's theorem.
	\end{note}

	It then follows that
	\begin{align}
		\int_\eta f(w) \dif w
		 & = \int_\eta f(z) + \psi(w) \dif w                    \nonumber \\
		 & = f(z) \int_\eta 1 \dif w + \int_\eta \psi(w) \dif w \nonumber \\
		 & = f(z) h + \int_\eta \psi(w) \dif w
		\label{eq:7}
	\end{align}
	Combining \eqref{eq:6} and \eqref{eq:7}, we have
	\begin{align}
		\abs{\frac{F(z+h) - F(z)}{h} - f(z)}
		 & = \frac{1}{\abs{h}} \abs{\int_\eta \psi(w) \dif w}            \nonumber  \\
		 & \text{Apply the ML inequality}                                 \nonumber \\
		 & \leq \frac{1}{\abs{h}} \max_{w \in \eta} \abs{\psi(w)} \abs{h} \nonumber \\
		 & = \max_{w \in \eta} \abs{\psi(w)}
		\label{eq:8}
	\end{align}
	As $h \to 0$, we have $w \to z$ and hence $\psi(w) \to 0$.
	Therefore, the limit of the right-hand side of \eqref{eq:8}
	is $0$ as $h \to 0$.
	This proves $F^\prime(z) = f(z)$.
\end{proof}

%------------------------------

\section{Cauchy's Integral Formula}

\begin{theorem}[Cauchy's Integral Formula] \label{thm:5}
	Let $D$ be an open disk,
	and $\Omega$ an open set containing $\overline{D}$.
	If $f$ is holomorphic in $\Omega$.
	Then for any point $z \in D$, we have
	\begin{align*}
		f(z) = \frac{1}{2 \pi i} \oint_{C} \frac{f(\zeta)}{\zeta - z} \dif \zeta
	\end{align*}
	where $C = \partial D$ is the positively oriented circle boundary of $D$.
\end{theorem}

\begin{proof}
	We will consider the contour integrals of the integrand
	\begin{align*}
		g(\zeta) = \frac{f(\zeta)}{\zeta - z}, \quad \zeta \in D \setminus \{z\}
	\end{align*}
	Consider the $\Gamma_{\delta, \varepsilon}$ be a keyhole contour illustrated
	in Figure~\ref{fig:2}.
	Here $\delta$ denotes width of the corridor and $\varepsilon$
	is the radius of the small arc $\gamma_{\varepsilon}$ centered at $z$.
	The outer arc is denoted by $\gamma$ and the two line segments forming the corridor
	are denoted by $\ell_1$ and $\ell_2$.
	Cauchy's Theorem tells us the counter integral
	along $\Gamma_{\delta, \varepsilon}$ is zero, i.e.,
	\begin{align}
		\int_\gamma g(\zeta) \dif \zeta
		+ \int_{\gamma_{\varepsilon}} g(\zeta) \dif \zeta
		+ \int_{\ell_1} g(\zeta) \dif \zeta
		+ \int_{\ell_2} g(\zeta) \dif \zeta
		= 0
		\label{eq:9}
	\end{align}

	\begin{figure}[H]
		\centering
		\includegraphics[width=0.7\textwidth]{figures/keyhole-for-proving-the-cauchy-integral-formula.png}
		\caption{The keyhole contour $\Gamma_{\delta, \varepsilon}$.}
		\label{fig:2}
	\end{figure}

	As $\delta \to 0$,
	the corridor gets narrower and narrower,
	the integrals along $\ell_1$ and $\ell_2$
	will cancel each other.
	To see this, consider the illustration in Figure~\ref{fig:3}.

	\begin{figure}[H]
		\centering
		\includegraphics[width=0.7\textwidth]{figures/adding-a-quadrilateral-to-two-line-segments.png}
		\caption{By adding a quadrilateral to the two line segments, the sum of the contour integrals along $\ell_1$ and $\ell_2$ is equivalently transformed to the sum of contour integrals along $\ell_3$ and $\ell_4$. As $\delta \to 0$, both lengths of $\ell_3$ and $\ell_4$ tend to zero, and hence the contour integrals along them will also vanish
			by the ML inequality.}
		\label{fig:3}
	\end{figure}

	Meanwhile, as $\delta \to 0$,
	the contour integrals along arcs will tend to
	the integrals along the circles, i.e.,
	$\int_\gamma g(\zeta) \dif \zeta \to \oint_C g(\zeta) \dif \zeta$
	and $\int_{\gamma_{\varepsilon}} g(\zeta) \dif \zeta \to \oint_{C_\varepsilon} g(\zeta) \dif \zeta$
	Hence, letting $\delta \to 0$, \eqref{eq:9} becomes
	\begin{align}
		\oint_C g(\zeta) \dif \zeta
		+ \oint_{C_\varepsilon} g(\zeta) \dif \zeta
		= 0
		\label{eq:10}
	\end{align}
	We now turn our focus to the contour integral on $C_\varepsilon$.
	Exploiting the fact that $f$, we can write
	\begin{align*}
		g(\zeta) & = \frac{f(\zeta)}{\zeta - z}      \\
		         & = \frac{1}{\zeta - z} (
		f(z) + f^\prime(z) (\zeta - z) + \psi(\zeta)
		)                                            \\
		         & = \frac{f(z)}{\zeta - z} + \left[
			f^\prime(z) + \frac{\psi(\zeta)}{\zeta - z}
			\right]
	\end{align*}
	where $\frac{\psi(\zeta)}{\zeta - z} \to 0$ as $\zeta \to z$.
	Note that the term $f^\prime(z) + \frac{\psi(\zeta)}{\zeta - z}$ is bounded
	in a sufficiently small neighborhood of $z$.
	Therefore, as $\varepsilon \to 0$, $\oint_{C_\varepsilon} f^\prime(\zeta) + \frac{\psi(\zeta)}{\zeta - z} \dif \zeta \to 0$.
	Hence, the only $\frac{f(z)}{\zeta - z}$ will contribute to
	the integral as $\varepsilon \to 0$.
	Let the negatively oriented
	circle $C_\varepsilon$ be the parametrized by $\zeta(t) = z + e^{i t}$
	where $t$ goes from $2\pi$ to $0$.
	We have
	\begin{align*}
		\oint_{C_\varepsilon} \frac{f(z)}{\zeta - z} \dif \zeta
		 & = f(z) \oint_{2 \pi}^{0} \frac{1}{e^{i t}} i e^{i t} \dif t \\
		 & = -f(z) \oint_{0}^{2 \pi} i \dif t                          \\
		 & = - 2 \pi i f(z)
	\end{align*}
	In summary,
	\begin{align}
		\lim_{\varepsilon \to 0} \oint_{C_\varepsilon} g(\zeta) \dif \zeta = - 2 \pi i f(z)
		\label{eq:11}
	\end{align}

	Letting $\varepsilon \to 0$ in \eqref{eq:10} and plugging in \eqref{eq:11},
	we obtain
	\begin{align*}
		\oint_C g(\zeta) \dif \zeta - 2 \pi i f(z) = 0
	\end{align*}
	which is exactly what we wanted to show.
\end{proof}

\begin{corollary}
	If $f$ is a holomorphic function in $\Omega$,
	then $f$ is complex differentiable infinitely many times
	in $\Omega$.
	Moreover, if $D \subseteq \Omega$
	is disk whose closure $\overline{D}$ is also contained in $\Omega$,
	then the $n$-th derivative of $f$ is given by
	\begin{align}
		f^{(n)}(z) = \frac{n!}{2 \pi i} \oint_C \frac{f(\zeta)}{(\zeta - z)^{n+1}} \dif \zeta, \quad \forall z \in D
		\label{eq:12}
	\end{align}
	where $C = \partial D$ is the positively oriented circle boundary of $D$.
\end{corollary}

\begin{proof}
	We shall prove by induction on $n$.
	The induction hypothesis is
	that $f$ has $n$-th derivative everywhere in $\Omega$
	and it is given by \eqref{eq:12} for all $n \in \N^\ast$.

	\noindent\textbf{Base Case:} The conclusion immediately follows
	from Theorem~\ref{thm:1} when $n=1$.

	\noindent\textbf{Inductive Step:} Assume \eqref{eq:12} holds
	for $n=k$. We will show that it also holds for $n=k+1$.
	By the induction hypothesis, now it is given that $f$ has $k$-th order derivative
	everywhere in $\Omega$ and the derivative is given by \eqref{eq:12}.

	Pick a closed disk $\overline{D} \subseteq \Omega$ and a point $z$ in it.
	By the definition of complex derivatives, we consider the quotient
	\begin{align}
		\frac{f^{(k)}(z+h) - f^{(k)}(z)}{h}
		= \frac{k!}{2 \pi i} \cdot \frac{1}{h} \oint_C f(\zeta) \left(
		\frac{1}{(\zeta - z - h)^{k+1}} - \frac{1}{(\zeta - z)^{k+1}}
		\right) \dif \zeta
		\label{eq:13}
	\end{align}
	Using the formula
	\begin{align*}
		A^m - B^m = (A - B) \sum_{j=0}^{m-1} A^{m-j-1} B^j
	\end{align*}
	the term in the integrand of \eqref{eq:13} can be simplified to
	\begin{align}
		\frac{1}{(\zeta - z - h)^{k+1}} - \frac{1}{(\zeta - z)^{k+1}}
		 & = \left( \frac{1}{\zeta - z - h} - \frac{1}{\zeta - z} \right)
		\sum_{j=0}^{k} \frac{1}{(\zeta - z - h)^{k-j} (\zeta - z)^j}                                          \nonumber \\
		 & = \frac{h}{(\zeta - z - h)(\zeta - z)} \sum_{j=0}^{k} \frac{1}{(\zeta - z - h)^{k-j} (\zeta - z)^j}
		\label{eq:14}
	\end{align}
	Substituting \eqref{eq:14} into \eqref{eq:13} gives
	\begin{align}
		\frac{f^{(k)}(z+h) - f^{(k)}(z)}{h}
		= \frac{k!}{2 \pi i} \oint_C f(\zeta)
		\frac{1}{(\zeta - z - h)(\zeta - z)} \sum_{j=0}^{k} \frac{1}{(\zeta - z - h)^{k-j} (\zeta - z)^j}
		\dif \zeta
		\label{eq:15}
	\end{align}
	Now, taking letting $h \to 0$ on both sides of \eqref{eq:15},
	we observe that $(\zeta - z - h) \to (\zeta - z)$.

	\begin{note}
		Here, we will interchange the order of $\lim_{h \to 0}$ and $\oint_C$.
		One should verify that this is indeed allowed and
		the limit on the right-hand side of \eqref{eq:15} exists.
		The approximation
		\begin{align*}
			\abs{\frac{1}{(z+h)^p} - \frac{1}{z^p}}
			\leq \abs{h} \frac{p 2^p}{\abs{z}^{p+1}},
			\quad p \in \N^\ast \text{ and $h$ is sufficiently small}
		\end{align*}
		maybe helpful.
	\end{note}

	Hence, \eqref{eq:15} yields
	\begin{align*}
		\lim_{h \to 0} \frac{f^{(k)}(z+h) - f^{(k)}(z)}{h}
		= \frac{k!}{2 \pi i} \oint_C f(\zeta)
		\frac{1}{(\zeta - z)^2} \frac{k+1}{(\zeta - z)^{k}}
		\dif \zeta
		= \frac{(k+1)!}{2 \pi i} \oint_C
		\frac{f(\zeta)}{(\zeta - z)^{k+2}}
		\dif \zeta
	\end{align*}
	which exactly \eqref{eq:12} when $n=k+1$.
	And we can show that $f$ has $k+1$-th order derivative everywhere in $\Omega$.
	This completes the proof.
\end{proof}


\begin{theorem} \label{thm:1}
	Suppose $f$ is holomorphic in an open set $\Omega$, and $D$ is an open disk
	centered at $z_0$ whose closure is contained in $\Omega$. Then $f$ has a power
	series expansion at $z_0$ in the form
	\begin{align}
		f(z) = \sum_{n=0}^\infty \frac{f^{(n)}(z_0)}{n!}  (z - z_0)^n \quad \forall z \in D
		\label{eq:1}
	\end{align}
\end{theorem}


\begin{theorem}[Liouville's Theorem] \label{thm:2}
	Let $f$ be an entire function.
\end{theorem}

\begin{theorem}[Fundamental Theorem of Algebra] \label{thm:3}
	Every polynomial with degree greater than $0$ with complex coefficients has a
	root in $\C$.
\end{theorem}

\begin{proof}
	Let polynomial $p(z)$ be given by
	\begin{align*}
		p(z) = a_n z^n + \cdots a_1 z + a_0
	\end{align*}
	where $n \geq 1$ and $a_n \neq 0$.
	We shall prove by contradiction.
	Assume $p(z)$ has no roots in $\C$. Then the reciprocal $1 / p(z)$ is defined on the entire
	complex plane.
	Moreover, the derivative of $1 / p(z)$ clearly exists everywhere.
	Therefore, $1 / p(z)$ is entire.
	We will use Liouville's theorem to establish a contradiction.
	To do so, we are going to show $1 / p(z)$ is bounded.

	We have
	\begin{align*}
		\frac{p(z)}{z^n} = a_n + \frac{a_{n-1}}{z} + \cdots + \frac{a_1}{z^{n-1}} + \frac{a_0}{z^n}
	\end{align*}
	Note that when $\abs{z} \to \infty$, the modulus of the quotient $p(z) / z^n$ will
	tend to $\abs{a_n}$.
	This implies that there exists $R > 0$ such that
	\begin{align}
		\abs{\frac{p(z)}{z^n}} > \frac{\abs{a_n}}{2}
		\label{eq:2}
	\end{align}
	whenever $\abs{z} > R$. Rearranging \eqref{eq:2}, we have
	\begin{align*}
		\abs{\frac{1}{p(z)}} < \frac{2}{\abs{a_n}} \cdot \frac{1}{\abs{z}^n} < \frac{2}{\abs{a_n} R^n} \quad \text{if } \abs{z} > R
	\end{align*}
	The above inequality shows $1 / p(z)$ is bounded outside the disk $D$ centered
	at $0$ with radius $R$.

	For points inside the closed disk $\overline{D}$, since function $\abs{1 / p(z)}$ is
	continuous and $\overline{D}$ is compact, $\abs{1 / p(z)}$ attains its maximum $M$ on $\overline{D}$.

	Therefore, we see that $1 / p(z)$ is indeed bounded on entire complex plane $\C$.
	By Liouville's theorem \ref{thm:2}, $1 / p(z)$ is constant, which leads to a
	contradiction since polynomials with degree greater than $0$ are non-constant.


	\begin{note}
		The last assertion that polynomials with degree greater than $0$ are
		non-constant seem evident. But the reader should still prove it. See Proposition~\ref{pro:1}.
		And the proof is not that trivial.
	\end{note}
\end{proof}

\begin{proposition} \label{pro:1}
	Polynomials with degree greater than $0$ are non-constant.
\end{proposition}

Now, we have proved polynomial $p(z)$ must have one root. Using mathematical
induction we can show that it actually has $n$ roots, and can be factorized as
\begin{align*}
	p(z) = a_n (z^n - z_1) (z^n - z_2) \cdots (z^n - z_n)
\end{align*}

\begin{corollary}
	Every polynomial $p(z) = a_n z^n + \cdots + a_1 z + a_0$ of degree $n \geq 1$ with
	complex coefficients has exactly $n$ roots in $\C$. If these $n$ roots are
	denoted by $z_1, \ldots, z_n$, then $p(z)$ can be factorized as
	\begin{align*}
		p(z) = a_n (z^n - z_1) (z^n - z_2) \cdots (z^n - z_n)
	\end{align*}
\end{corollary}

\begin{proof}
	We shall prove by induction.

	\noindent\textbf{Base Case:} Suppose $n = 1$, then $p(z) = a_1 z + a_0$ ($a_1 \neq 0$). Setting $p(z) = 0$, we solve that $z = -a_0 / a_1$. Therefore, $z_1 = -a_0 / a_1$ is the only root of $p(z)$, and we can write $p(z) = a_1(z - z_1)$.

	\noindent\textbf{Inductive Step:} Assume this corollary holds for $n = k$, we need to show that it also holds for $n = k + 1$.
	By the Fundamental Theorem of Algebra \ref{thm:3}, there exists one root $z_1$ for $p(z)$.
	We want to show that $p(z)$ has a factor $(z - z_1)$.
	Let $\tilde{p}(z) = p(z + z_1)$.
	Note that $\tilde{p}(z)$ is also a polynomial of degree $k + 1$.
	Write
	\begin{align}
		\tilde{p}(z) = b_{k+1} z^{k+1} + \cdots + b_1 z + b_0
		\label{eq:3}
	\end{align}
	Because $z_1$ is a root of $p(z)$, we have $\tilde{p}(0) = p(z_1) = 0$. Substituting $z=0$ into \eqref{eq:3} yields $b_0 = 0$.
	Therefore, we can write
	\begin{align*}
		\tilde{p}(z) = z (b_{k+1} z^{k} + \cdots b_2 z + b_1)
	\end{align*}
	Then, we recover the expression of $p(z)$ from $\tilde{p}(z)$.
	Note that $p(z) = \tilde{p}(z - z_1)$.
	Hence,
	\begin{align*}
		p(z) = (z - z_1) \underbrace{(b_{k+1} z^{k} + \cdots b_2 z + b_1)}_{\text{a polynomial of degree $k$}}
		= (z - z_1) q(z)
	\end{align*}
	Now, we may apply the induction hypothesis on $q(z)$.
	We have
	\begin{align*}
		p(z) & = (z - z_1) [c (z - z_2) \cdots (z - z_{k+1})] \\
		     & = c (z - z_1) \cdots (z - z_{k+1})
	\end{align*}
	And we realized that $c$ is exactly $a_{k+1}$ since it is the coefficient of $z^{k+1}$.
	Therefore, we have shown
	\begin{align}
		p(z) = a_{k+1} (z - z_1) \cdots (z - z_{k+1})
		\label{eq:4}
	\end{align}
	Equation~\eqref{eq:4} tells us that $z_1, \ldots, z_{k+1}$ are roots of $p(z)$.
	Moreover, $p(z)$ cannot have any other roots. To see this, suppose $w \neq z_j \ \forall j=1, \ldots, k+1$. Then, substituting $z = w$ in \eqref{eq:4}, we see that $p(w) \neq 0$ since each factor is nonzero.
\end{proof}


The next theorem roughly demonstrates that the global behaviour of a holomorphic function is determined by its values on an appropriate small subset.

\begin{theorem}
	If $f$ is a holomorphic function in a connected open set $\Omega$, and vanishes on a sequence of distinct points whose limit is also in $\Omega$, then $f$ is constantly zero in $\Omega$.

	In other words, let $f$ be a holomorphic function in a connected open set $\Omega$.
	Suppose $\{z_n\}_{n \in \Z^+}$ is a sequence of distinct points in $\Omega$, and it converges to $z_0 \in \Omega$. If $f(z_n) = 0$ for all $n$ and $f(z_0) = 0$, then $f(z) = 0$ for all $z \in \Omega$.
\end{theorem}

At a first glance, this result seems magical and almost unrealistic.
How can the value of a function vanishes in the entire set only because it vanishes on a sequence of distinct points?


\begin{proof}
	We will first show that $f$ vanishes in an open disk $D_r(z_0)$ of $z_0$.
	Since $f$ is holomorphic in $\Omega$, it is given by its Taylor series expansion at $z_0$:
	\begin{align*}
		f(z) = \sum_{n=0}^\infty a_n  (z - z_0)^n \quad \forall z \in D_r(z_0)
	\end{align*}
	Assume $f$ is not constantly zero.
	Then we can find the first nonzero coefficient $a_m$ in the expansion.
	Write
	\begin{align}
		f(z) & = a_m(z - z_0)^m + \sum_{k=1}^\infty a_{m+k}  (z - z_0)^{m+k} \nonumber \\
		     & = a_m(z - z_0)^m \left[
			1 + \sum_{k=1}^\infty \frac{a_{m+k}}{a_m}  (z - z_0)^{k}
			\right]
		\label{eq:5}
	\end{align}
	Note that the term $\sum_{k=1}^\infty \frac{a_{m+k}}{a_m}  (z - z_0)^{k}$ is a continuous function that vanishes at $z_0$.
	Hence, we may find a sufficiently small neighbourhood $N$ of $z_0$ such that $1 + \sum_{k=1}^\infty \frac{a_{m+k}}{a_m}  (z - z_0)^{k}$ is nonzero in $N$.
	By the given condition, there exists a point $z_j$ in the sequence such that $z_j \neq z_0$ and $z_j \in N$.
	Plugging $z_j$ into \eqref{eq:5}, we have
	\begin{align*}
		0 = f(z_j) = a_m(z_j - z_0)^m \left[
			1 + \sum_{k=1}^\infty \frac{a_{m+k}}{a_m}  (z_j - z_0)^{k}
			\right]
	\end{align*}
	This leads to a contradiction since the right-hand side is nonzero.
	Therefore, $f$ is constantly zero in the disk $D_r(z_0)$.

	Next, we will show that $f$ is zero in the entire set $\Omega$ by exploiting the fact that $\Omega$ is connected.
	Let set $U$ be defined as the interior of the set of all points at which $f$ vanishes, i.e., $U = [f^{-1}(\{0\})]^\circ$
	\begin{note}
		We apply the interior here to enforce that $U$ is open. Now, we will show that it is also closed and hence $U = \Omega$. (Of course, $U \neq \emptyset$.)
	\end{note}

	Note that $U$ is a nonempty open subset of $\Omega$.
	Let $w$ be an accumulation point of $U$.
	Then we can find a sequence $\{w_n\}$ of distinct points in $U$.
	And by the definition of $U$, $f(w_n) = 0 \; \forall n$.
	Applying what we have proved above, we conclude that $f(z) = 0 \; \forall z \in D_{r^\prime} (w)$.
	This shows that $w \in U$.
	Therefore, $U$ contains all its points of accumulation, which means it is also closed.
	Because $\Omega$ is connected and $U$ is a nonempty, both open and closed subset in $\Omega$, it follows that $U = \Omega$.
	Therefore, $f(z) = 0 \; \forall z \in \Omega$.
\end{proof}

%------------------------------


%==============================

% references
\printbibliography[heading=bibintoc, title=References]

%==============================

% print index page
\printindex

\end{document}
